%! TeX program = lualatex
\documentclass[
    11pt,
    a4paper,
    oneside, % voor enkel digitale output, als je print en bind is 2-zijdig allicht beter
    % hidelinks, % links zijn clickable in pdf maar dat wordt niet getoond door kleur of zo
    dutch,
    ]{memoir}

\usepackage{fontenc}
\usepackage{microtype}
\usepackage{graphicx}
\usepackage{wrapfig}
\usepackage{xcolor}
\usepackage{babel}
\usepackage{csquotes}

\usepackage[blankBeforeHeading, html, fencedCode,
            citations, definitionLists,
            hashEnumerators, smartEllipses, hybrid, pipeTables]{markdown}

\usepackage{hyperref}
\usepackage{cleveref}  % \cref of \Cref voor 'tabel 2.1' of 'Figuur 2.3'
\usepackage{float}     % voor [H] float option om 'here' te forceren
\usepackage{flafter}   % plaats floats nooit voor punt waar ze zijn gedeclareerd
\usepackage{fontspec}  % Voor gebruik van opentype fonts enz.
\usepackage{xspace}

% fontpackage
\usepackage{lmodern} % de default
% \usepackage{libertine}
% \usepackage{times}
% \usepackage{noto}

% Vervanging van font shapes die niet bestaan in lmodern
\DeclareFontShape{TU}{lmr}{bx}{sc}{<->ssub*lmr/m/sc}{}
\DeclareFontShape{TU}{lmss}{bx}{sc}{<->ssub*lmss/bx/n}{}

\directlua{require("mystyle/functions.lua")}
\usepackage{mystyle/colors}
\usepackage{mystyle/citations}
\usepackage{mystyle/quotations}
\usepackage{mystyle/chapters}

\setlrmarginsandblock{2.5cm}{2.5cm}{*} % bij oneside en niet printen/binden is links 2.5cm allicht beter
\setulmarginsandblock{2.5cm}{*}{1}
\checkandfixthelayout

\OnehalfSpacing

% Beïnvloed de ruimte tussen alle listitems in het document. 
% Wanneer er per-lijst controlle nodig is zet dit dan na begin van de list zelf
% \firmlists % medium ruimte tussen list items
% \tightlists % weinig ruimte tussen list items

% Bronvermelding bij bijvoorbeeld figuren of tabellen
\newcommand{\source}[1]{\vspace{-5pt}Bron: {#1} }

\newcommand{\vraag}[1]{\vspace{5pt}\noindent\textit{#1}\vspace{0pt}\\}  % geknoei met vspace omdat pictureparagraph even te lastig is om goed te krijgen
\newcommand{\stelling}[2]{\paragraph{#1} \textit{#2}\vspace{5pt}\\}

% Mijn eerste lua macro, voor het netjes neerzetten van acronyms met small caps
\newcommand{\afko}[1]{\directlua{acronym("#1")}}

% iets mooiere %, lijkt beter wanneer dit direct volgt op een getal
\newcommand*\pct{\scalebox{0.8}{\%}}

% \newfontface{\remarkableFont}[Path = ./fonts/]{reMarkablePortrait-Medium-Web.otf}
% \newcommand{\remarkable}[0]{{\remarkableFont reMarkable}\xspace}
% \newcommand{\remarkable}[0]{reMarkable\xspace}


% https://tex.stackexchange.com/questions/619760/put-picture-next-to-paragraph
%Create a new minipage environment where paragraphs have indents
\newlength{\currentparindent}
\newenvironment{minipageparindent}[2][c]
  {\setlength{\currentparindent}{\parindent}% save the value
   \noindent\begin{minipage}[#1]{#2}% open the minipage
   \setlength{\parindent}{\currentparindent}% restore the value
  }
  {\end{minipage}}

%Create an environment that creates a paragraph with a picture next to it, both being aligned at the top.
\newenvironment{pictureparagraph}[1]    %argument is an 'includegraphics' command
    {\newcommand\picturetoplace{#1}     % using variable directly gives an error
    \begin{minipageparindent}[t]{0.7\textwidth} %
    \vspace{0pt}    %Make sure that the top base is at the absolute top
    }   
    {
    \end{minipageparindent}
    \begin{minipage}[t]{0.3\textwidth}
    \vspace{0pt}    %Make sure that the top base is at the absolute top of the minipage
    \picturetoplace
    \end{minipage}\\}

\begin{document}

\thispagestyle{empty}
{
\sffamily
\centering
\Large

~\vspace{\fill}

{\huge\color{TitleColor}
Plan van aanpak 
}

\vspace{2.5cm}

{\LARGE
Rykle Baron
}

\vspace{3.5cm}

Studentnummer: s1150150 \\
Klas: ltb-ict \\
Docent: Alwin Truin  \\
Opleiding: Docent 2\textsuperscript{e} graad techniek ict \\
Module Code: LEBOVO.PROF.X.23\\
Module naam: Profilering Beroep: Onderwijsvernieuwing en curriculumontwerp \\


\vspace{3.5cm}

% hspace zodat de naam van het logo gecentreerd is, niet het hele logo
\hspace{13mm}\includegraphics[width=6cm]{images/windesheim.pdf}

\vspace{\fill}

Maart 2025

}
\cleardoublepage


\tableofcontents*

\clearpage


\chapter{Inleiding}
# Aanleiding

Het team waar ik deel van uitmaak, staat voor een grote verandering in de manier waarop wij onderwijs vormgeven. We stappen af van losse vakken en gaan volledig over op projectmatig onderwijs. Deze transitie moet binnen een jaar afgerond zijn. Voor de implementatie van deze verandering is er per docent slechts twee uur per week beschikbaar, wat leidt tot een hoge werkdruk en merkbare stress binnen het team.

Door deze beperkte tijd richten we ons vooral op het vertalen van de bestaande vakken naar de nieuwe projecten. Tegelijkertijd krijgen we volgend schooljaar te maken met een nieuw kwalificatiedossier, maar door de hoge werkdruk blijft dit momenteel op de achtergrond. Dit baart mij zorgen, omdat er wel degelijk wijzigingen in het dossier staan die essentieel zijn voor de studenten. Zij hebben deze aanpassingen nodig binnen de projecten om goed voorbereid te zijn op hun examens.

Met mijn beroepsproduct wil ik ervoor zorgen dat de veranderingen uit het nieuwe kwalificatiedossier een plek krijgen binnen de projecten. Mijn aanpak bestaat uit drie stappen:

    Analyse – Ik onderzoek zowel het huidige als het nieuwe kwalificatiedossier en breng de veranderingen in kaart.
    Koppeling – Ik verbind deze veranderingen aan de nieuwe projecten: waar passen ze binnen ons onderwijs?
    Stroomschema – Ik ontwikkel een overzichtelijk schema dat helpt om stapsgewijs te bepalen of en hoe de veranderingen zijn geïntegreerd in de projecten.

Het stroomschema werkt als een beslisboom. Je begint bij het eerste vakje en bepaalt of aan de gestelde eis wordt voldaan. Zo ja, dan volg je de pijl naar de volgende stap. Zo nee, dan leidt een alternatieve route naar de benodigde aanpassingen. Uiteindelijk doorloop je het schema totdat alle eisen zijn verwerkt en de verandering is opgenomen in het project.

Met dit beroepsproduct wil ik een praktisch hulpmiddel ontwikkelen dat niet alleen nu, maar ook in de toekomst houvast biedt bij curriculumwijzigingen.

 
De onderzoeksvraag concreet: 
\begin{displayquote}    
    Om de overstap naar projectmatig onderwijs soepel te laten verlopen, ontwikkel ik een stroomschema dat ervoor zorgt dat de veranderingen uit het nieuwe kwalificatiedossier op de juiste plek binnen de projecten terechtkomen. Zo waarborg ik dat studenten goed voorbereid zijn op hun examens. 
\end{displayquote}


\subsection{swot analyze}
\subsubsection{projectonderwijs}
# SWOT-analyse – Beroepsproduct: Integratie van het kwalificatiedossier in projectonderwijs

## Sterktes (Strengths)
- Ik heb een goed analytisch vermogen, waardoor ik de veranderingen in het nieuwe kwalificatiedossier kan doorgronden en vertalen naar de praktijk.
- Ik ben probleemoplossend ingesteld en zie kansen om het kwalificatiedossier op een gestructureerde manier te koppelen aan de nieuwe projectvormen.
- Ik heb inzicht in het belang van het kwalificatiedossier en begrijp dat deze wijzigingen cruciaal zijn voor studenten om goed voorbereid hun examens in te gaan.
- Ik ben betrokken bij de onderwijsontwikkeling en neem het initiatief om ervoor te zorgen dat de overgang naar projectonderwijs niet ten koste gaat van de kwaliteit en examineringseisen.

## Zwaktes (Weaknesses)
- Mijn ervaring met curriculumontwikkeling en het verwerken van kwalificatiedossiers in een nieuw onderwijssysteem is nog beperkt.
- Het maken van de vertaalslag van theorie naar praktijk is uitdagend, vooral omdat ik de juiste koppelingen moet maken tussen het kwalificatiedossier en de projecten.
- De tijd die ik heb om dit beroepsproduct te ontwikkelen is beperkt tot twee uur per week, wat kan betekenen dat ik keuzes moet maken in de uitvoerbaarheid.
- Mijn kennis van de generieke kennisbasis en de manier waarop kwalificatiedossiers officieel verwerkt worden in het onderwijs kan nog verder ontwikkeld worden.

## Kansen (Opportunities)}
- Door dit beroepsproduct te ontwikkelen, verbeter ik mijn kennis en vaardigheden op het gebied van curriculumontwikkeling en onderwijsinnovatie. Dit draagt bij aan mijn professionele groei.
- Mijn beroepsproduct kan niet alleen mij helpen, maar ook mijn collega’s ondersteunen bij de implementatie van projectmatig onderwijs en de verwerking van het kwalificatiedossier.
- Door de veranderingen uit het kwalificatiedossier goed te verwerken, help ik studenten beter voorbereid hun examens in te gaan.
- Ik kan samenwerken met collega’s en mogelijk externe experts om mijn beroepsproduct te verbeteren en beter aan te laten sluiten op de onderwijspraktijk.

## Bedreigingen (Threats)
- Er kan weerstand zijn binnen het team tegen de veranderingen in het onderwijs en het kwalificatiedossier, wat de implementatie van mijn beroepsproduct lastiger maakt.
- Als het kwalificatiedossier niet duidelijk genoeg is geformuleerd of onvoldoende gecommuniceerd wordt, kan dat problemen opleveren bij de verwerking ervan in de projecten.
- De beperkte tijd en hoge werkdruk maken het een uitdaging om een volledig uitgewerkt en goed getest beroepsproduct op te leveren binnen de gestelde termijn.
- Het stroomschema dat ik ontwikkel moet mogelijk in de toekomst aangepast worden, omdat onderwijseisen en kwalificatiedossiers blijven veranderen.

\subsection{Stap 2. Verlegenheidssituatie en onderwerp}


Mijn handelingsverlegenheid komt voort uit verschillende factoren die te maken hebben met de complexiteit van het kwalificatiedossier en het ontwikkelen van een beslisboom, iets waar ik geen ervaring mee heb. De grootte en de gedetailleerdheid van het kwalificatiedossier maken het lastig om alles te doorgronden en de juiste koppelingen te maken met de nieuwe projectstructuur. Dit vraagt veel tijd en aandacht, en ik twijfel of ik het proces goed en op tijd kan afronden.

Daarnaast is het maken van de beslisboom zelf een nieuwe uitdaging voor mij. Ik heb nog geen ervaring met het ontwerpen van een dergelijk hulpmiddel, en ik vraag me af of het daadwerkelijk het gewenste resultaat zal opleveren. De structuur van de beslisboom, met de verschillende voorwaarden en doorverwijzingen, is complex, en ik ben onzeker over hoe ik deze effectief moet opstellen zonder dat het te ingewikkeld wordt.

Een andere bron van handelingsverlegenheid is de onzekerheid over hoe mijn collega's zullen reageren op de beslisboom. Het idee om een stroomschema of beslisboom te introduceren kan voor hen als betuttelend of hiërarchisch overkomen, aangezien het een voorschrijvende manier van werken is. Ik ben bang dat sommige collega's het als een dwingende structuur kunnen ervaren, wat mogelijk weerstand oproept. Dit maakt het voor mij moeilijk om te bepalen hoe ik het beroepsproduct op een manier kan presenteren die zowel effectief is als goed ontvangen wordt door het team.


\subsubsection{Specifiek}
Ik wil na het afronden van deze profilering een volledig functionerende beslisboom hebben ontwikkeld die de veranderingen uit het nieuwe kwalificatiedossier op een gestructureerde manier koppelt aan de nieuwe projectvormen. De beslisboom moet duidelijk en gebruiksvriendelijk zijn voor mijn collega's, zodat zij deze kunnen volgen bij het ontwerpen en uitvoeren van hun projecten.

\subsubsection{Meetbaar}
Het succes van het doel wordt gemeten aan de hand van de voltooiing van de beslisboom en de mate waarin mijn collega's deze kunnen gebruiken zonder verdere uitleg. Ik wil dat minimaal 80\% van mijn collega's de beslisboom zonder problemen kunnen gebruiken bij hun onderwijspraktijk.

\subsubsubsection{Acceptabel}
Dit doel is acceptabel omdat het aansluit bij de veranderingen die we als team doormaken (de overstap naar projectonderwijs en het implementeren van het nieuwe kwalificatiedossier). Het is ook haalbaar gezien de twee uur per week die ik heb om aan dit project te werken, hoewel er beperkingen zijn qua tijdsdruk.

\subsubsubsection{Realistisch}
Het is realistisch om een beslisboom te ontwikkelen die de veranderingen in het kwalificatiedossier verwerkt, hoewel het een uitdaging zal zijn om de verschillende elementen van het dossier goed in kaart te brengen. Door gestructureerd te werken en mijn tijd goed in te delen, is het haalbaar om dit doel binnen de gestelde tijd te bereiken.

\subsubsubsection{Tijdgebonden}
Het doel is om de beslisboom te ontwikkelen en te implementeren binnen dit semester, zodat deze volledig operationeel is voor de afronding van het semester.

\subsubsection{Onderbouwing}
Door mijn bekwaamheid te ontwikkelen, zorg ik ervoor dat studenten die het nieuwe kwalificatiedossier gaan volgen, geen onverwachte onderwerpen tegenkomen op hun examen. De nieuwe onderwerpen komen namelijk aan bod in de projecten die zij uitvoeren. Het is essentieel voor mij als docent om goed op de hoogte te zijn van de veranderingen in het kwalificatiedossier ten opzichte van het oude dossier. Deze kennis kan ik direct toepassen bij de ontwikkeling van de nieuwe projecten. Door deze veranderingen grondig te begrijpen, ben ik beter voorbereid en loop ik niet achter de feiten aan, maar voorop. Als ik deze bekwaamheid goed ontwikkel, zal mijn beroepsproduct misschien zelfs overbodig worden, omdat ik de benodigde kennis dan al heb.

Mijn school, Noorderpoort, heeft veel baat bij de waarborging van de kwaliteit van het onderwijs, vooral met het oog op juridische zaken, zoals de onaangekondigde audits van de onderwijsinspectie. Als zo'n audit plaatsvindt, kunnen we precies aantonen hoe en waar we het nieuwe kwalificatiedossier in de lessen hebben geïmplementeerd. Door een tool zoals een stroomschema te ontwikkelen op basis van een grondige analyse van het kwalificatiedossier, laat ik zien dat het onderwijsteam professioneel en goed voorbereid te werk gaat.

De veranderingen in het kwalificatiedossier zijn niet zonder reden doorgevoerd. Ze zorgen ervoor dat in de ICT-branche de juiste ontwikkelingen plaatsvinden, die aansluiten bij de vraag van het bedrijfsleven. Het kwalificatiedossier wordt in samenwerking met het bedrijfsleven ontwikkeld, en door deze tool te maken, zorg ik ervoor dat de nieuwe behoeften van het bedrijfsleven in ons curriculum worden geïntegreerd. Dit stelt de studenten in staat om na hun diploma makkelijker aansluiting te vinden bij het bedrijfsleven, omdat zij de kennis en vaardigheden hebben opgedaan die daar worden gevraagd. Kortom, dit leidt tot betere werkgelegenheidsperspectieven voor studenten, wat hen in staat stelt om een stabiel inkomen te verdienen, belasting te betalen en bij te dragen aan de maatschappij.

\section{Stap 3. bekwaming}
Gedurende de transitie van lessen en projecten hebben wij als onderwijsteam ondersteuning van onderwijskundigen. Deze onderwijskundigen wil ik betrekken bij mijn profilering. Ik vraag mij op dit moment eigenlijk ineens af, welke tools zijn er om te waarborgen dat de juiste onderwerpen een plek krijgen binnen het curriculum. Door de onderwijskundigen te betrekken maak ik gebruik van experts. Daarnaast wil ik mij richten op het gebruik van de beslisboom of beter gezegd, hoe zorg ik ervoor dat mijn collega's dit gaan gebruiken. Daarom richt ik mij op literatuur over veranderingsmanagement {bron "Leiderschap bij verandering" – John P. Kotter}. 

% {\raggedright\printbibliography}
    
\appendix

\input{appendices}

\end{document}

