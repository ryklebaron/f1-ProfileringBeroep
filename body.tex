\chapter{Inleiding}
# Aanleiding

Het team waar ik deel van uitmaak, staat voor een grote verandering in de manier waarop wij onderwijs vormgeven. We stappen af van losse vakken en gaan volledig over op projectmatig onderwijs. Deze transitie moet binnen een jaar afgerond zijn. Voor de implementatie van deze verandering is er per docent slechts twee uur per week beschikbaar, wat leidt tot een hoge werkdruk en merkbare stress binnen het team.

Door deze beperkte tijd richten we ons vooral op het vertalen van de bestaande vakken naar de nieuwe projecten. Tegelijkertijd krijgen we volgend schooljaar te maken met een nieuw kwalificatiedossier, maar door de hoge werkdruk blijft dit momenteel op de achtergrond. Dit baart mij zorgen, omdat er wel degelijk wijzigingen in het dossier staan die essentieel zijn voor de studenten. Zij hebben deze aanpassingen nodig binnen de projecten om goed voorbereid te zijn op hun examens.

Met mijn beroepsproduct wil ik ervoor zorgen dat de veranderingen uit het nieuwe kwalificatiedossier een plek krijgen binnen de projecten. Mijn aanpak bestaat uit drie stappen:

    Analyse – Ik onderzoek zowel het huidige als het nieuwe kwalificatiedossier en breng de veranderingen in kaart.
    Koppeling – Ik verbind deze veranderingen aan de nieuwe projecten: waar passen ze binnen ons onderwijs?
    Stroomschema – Ik ontwikkel een overzichtelijk schema dat helpt om stapsgewijs te bepalen of en hoe de veranderingen zijn geïntegreerd in de projecten.

Het stroomschema werkt als een beslisboom. Je begint bij het eerste vakje en bepaalt of aan de gestelde eis wordt voldaan. Zo ja, dan volg je de pijl naar de volgende stap. Zo nee, dan leidt een alternatieve route naar de benodigde aanpassingen. Uiteindelijk doorloop je het schema totdat alle eisen zijn verwerkt en de verandering is opgenomen in het project.

Met dit beroepsproduct wil ik een praktisch hulpmiddel ontwikkelen dat niet alleen nu, maar ook in de toekomst houvast biedt bij curriculumwijzigingen.

 
De onderzoeksvraag concreet: 
\begin{displayquote}    
    Om de overstap naar projectmatig onderwijs soepel te laten verlopen, ontwikkel ik een stroomschema dat ervoor zorgt dat de veranderingen uit het nieuwe kwalificatiedossier op de juiste plek binnen de projecten terechtkomen. Zo waarborg ik dat studenten goed voorbereid zijn op hun examens. 
\end{displayquote}


\subsection{swot analyze}
\subsubsection{projectonderwijs}
# SWOT-analyse – Beroepsproduct: Integratie van het kwalificatiedossier in projectonderwijs

## Sterktes (Strengths)
- Ik heb een goed analytisch vermogen, waardoor ik de veranderingen in het nieuwe kwalificatiedossier kan doorgronden en vertalen naar de praktijk.
- Ik ben probleemoplossend ingesteld en zie kansen om het kwalificatiedossier op een gestructureerde manier te koppelen aan de nieuwe projectvormen.
- Ik heb inzicht in het belang van het kwalificatiedossier en begrijp dat deze wijzigingen cruciaal zijn voor studenten om goed voorbereid hun examens in te gaan.
- Ik ben betrokken bij de onderwijsontwikkeling en neem het initiatief om ervoor te zorgen dat de overgang naar projectonderwijs niet ten koste gaat van de kwaliteit en examineringseisen.

## Zwaktes (Weaknesses)
- Mijn ervaring met curriculumontwikkeling en het verwerken van kwalificatiedossiers in een nieuw onderwijssysteem is nog beperkt.
- Het maken van de vertaalslag van theorie naar praktijk is uitdagend, vooral omdat ik de juiste koppelingen moet maken tussen het kwalificatiedossier en de projecten.
- De tijd die ik heb om dit beroepsproduct te ontwikkelen is beperkt tot twee uur per week, wat kan betekenen dat ik keuzes moet maken in de uitvoerbaarheid.
- Mijn kennis van de generieke kennisbasis en de manier waarop kwalificatiedossiers officieel verwerkt worden in het onderwijs kan nog verder ontwikkeld worden.

## Kansen (Opportunities)}
- Door dit beroepsproduct te ontwikkelen, verbeter ik mijn kennis en vaardigheden op het gebied van curriculumontwikkeling en onderwijsinnovatie. Dit draagt bij aan mijn professionele groei.
- Mijn beroepsproduct kan niet alleen mij helpen, maar ook mijn collega’s ondersteunen bij de implementatie van projectmatig onderwijs en de verwerking van het kwalificatiedossier.
- Door de veranderingen uit het kwalificatiedossier goed te verwerken, help ik studenten beter voorbereid hun examens in te gaan.
- Ik kan samenwerken met collega’s en mogelijk externe experts om mijn beroepsproduct te verbeteren en beter aan te laten sluiten op de onderwijspraktijk.

## Bedreigingen (Threats)
- Er kan weerstand zijn binnen het team tegen de veranderingen in het onderwijs en het kwalificatiedossier, wat de implementatie van mijn beroepsproduct lastiger maakt.
- Als het kwalificatiedossier niet duidelijk genoeg is geformuleerd of onvoldoende gecommuniceerd wordt, kan dat problemen opleveren bij de verwerking ervan in de projecten.
- De beperkte tijd en hoge werkdruk maken het een uitdaging om een volledig uitgewerkt en goed getest beroepsproduct op te leveren binnen de gestelde termijn.
- Het stroomschema dat ik ontwikkel moet mogelijk in de toekomst aangepast worden, omdat onderwijseisen en kwalificatiedossiers blijven veranderen.

\subsection{Stap 2. Verlegenheidssituatie en onderwerp}


Mijn handelingsverlegenheid komt voort uit verschillende factoren die te maken hebben met de complexiteit van het kwalificatiedossier en het ontwikkelen van een beslisboom, iets waar ik geen ervaring mee heb. De grootte en de gedetailleerdheid van het kwalificatiedossier maken het lastig om alles te doorgronden en de juiste koppelingen te maken met de nieuwe projectstructuur. Dit vraagt veel tijd en aandacht, en ik twijfel of ik het proces goed en op tijd kan afronden.

Daarnaast is het maken van de beslisboom zelf een nieuwe uitdaging voor mij. Ik heb nog geen ervaring met het ontwerpen van een dergelijk hulpmiddel, en ik vraag me af of het daadwerkelijk het gewenste resultaat zal opleveren. De structuur van de beslisboom, met de verschillende voorwaarden en doorverwijzingen, is complex, en ik ben onzeker over hoe ik deze effectief moet opstellen zonder dat het te ingewikkeld wordt.

Een andere bron van handelingsverlegenheid is de onzekerheid over hoe mijn collega's zullen reageren op de beslisboom. Het idee om een stroomschema of beslisboom te introduceren kan voor hen als betuttelend of hiërarchisch overkomen, aangezien het een voorschrijvende manier van werken is. Ik ben bang dat sommige collega's het als een dwingende structuur kunnen ervaren, wat mogelijk weerstand oproept. Dit maakt het voor mij moeilijk om te bepalen hoe ik het beroepsproduct op een manier kan presenteren die zowel effectief is als goed ontvangen wordt door het team.


\subsubsection{Specifiek}
Ik wil na het afronden van deze profilering een volledig functionerende beslisboom hebben ontwikkeld die de veranderingen uit het nieuwe kwalificatiedossier op een gestructureerde manier koppelt aan de nieuwe projectvormen. De beslisboom moet duidelijk en gebruiksvriendelijk zijn voor mijn collega's, zodat zij deze kunnen volgen bij het ontwerpen en uitvoeren van hun projecten.

\subsubsection{Meetbaar}
Het succes van het doel wordt gemeten aan de hand van de voltooiing van de beslisboom en de mate waarin mijn collega's deze kunnen gebruiken zonder verdere uitleg. Ik wil dat minimaal 80\% van mijn collega's de beslisboom zonder problemen kunnen gebruiken bij hun onderwijspraktijk.

\subsubsubsection{Acceptabel}
Dit doel is acceptabel omdat het aansluit bij de veranderingen die we als team doormaken (de overstap naar projectonderwijs en het implementeren van het nieuwe kwalificatiedossier). Het is ook haalbaar gezien de twee uur per week die ik heb om aan dit project te werken, hoewel er beperkingen zijn qua tijdsdruk.

\subsubsubsection{Realistisch}
Het is realistisch om een beslisboom te ontwikkelen die de veranderingen in het kwalificatiedossier verwerkt, hoewel het een uitdaging zal zijn om de verschillende elementen van het dossier goed in kaart te brengen. Door gestructureerd te werken en mijn tijd goed in te delen, is het haalbaar om dit doel binnen de gestelde tijd te bereiken.

\subsubsubsection{Tijdgebonden}
Het doel is om de beslisboom te ontwikkelen en te implementeren binnen dit semester, zodat deze volledig operationeel is voor de afronding van het semester.

\subsubsection{Onderbouwing}
Door mijn bekwaamheid te ontwikkelen, zorg ik ervoor dat studenten die het nieuwe kwalificatiedossier volgen, geen onverwachte onderwerpen tegenkomen op hun examen. De nieuwe onderwerpen worden geïntegreerd in de projecten die zij uitvoeren. Als docent is het voor mij essentieel om goed op de hoogte te zijn van de veranderingen ten opzichte van het vorige kwalificatiedossier. Deze kennis pas ik direct toe bij de ontwikkeling van nieuwe projecten. Door deze veranderingen grondig te begrijpen, blijf ik niet achter de feiten aanlopen, maar neem ik juist een voortrekkersrol in. Wanneer ik deze bekwaamheid volledig ontwikkel, zou mijn beroepsproduct zelfs overbodig kunnen worden, omdat ik de benodigde kennis dan al verankerd heb.

Mijn school, Noorderpoort, heeft er belang bij dat de onderwijskwaliteit geborgd blijft, met name met het oog op juridische aspecten, zoals onaangekondigde audits van de onderwijsinspectie. Bij zo’n audit kunnen we aantonen hoe en waar het nieuwe kwalificatiedossier in de lessen is geïmplementeerd. Door een beslisboom te ontwikkelen op basis van een grondige analyse van het kwalificatiedossier, laat ik zien dat het onderwijsteam professioneel en gestructureerd te werk gaat.

De wijzigingen in het kwalificatiedossier zijn niet zonder reden doorgevoerd. Ze zorgen ervoor dat de ICT-opleiding blijft aansluiten op de actuele behoeften van het werkveld. Het kwalificatiedossier wordt mede in samenwerking met het bedrijfsleven ontwikkeld. Met mijn beroepsproduct zorg ik ervoor dat deze nieuwe eisen op een efficiënte en overzichtelijke manier worden verwerkt in het curriculum. Dit vergroot de kans dat afgestudeerden beschikken over relevante kennis en vaardigheden, waardoor zij beter voorbereid zijn op de arbeidsmarkt. Hierdoor dragen zij niet alleen bij aan de economie, maar ook aan de maatschappij.

\section{Stap 3. bekwaming}
Tijdens de transitie naar een projectmatige onderwijsvorm werken wij als onderwijsteam samen met onderwijskundigen. Ik wil hen betrekken bij mijn profilering om te onderzoeken welke tools kunnen helpen bij het waarborgen dat de juiste onderwerpen een plek krijgen binnen het curriculum. Dit stelt mij in staat om gebruik te maken van de expertise van specialisten.

Daarnaast richt ik mij op de implementatie van de beslisboom binnen het team. Het succes van mijn beroepsproduct hangt niet alleen af van de inhoud, maar ook van de acceptatie door collega’s. Daarom verdiep ik mij in literatuur over veranderingsmanagement, zoals Leiderschap bij verandering van John P. Kotter.

\section{Stap 4. Beroepsproduct}
Aan het einde van deze module heb ik een beslisboom ontwikkeld die ervoor zorgt dat de veranderingen in het nieuwe kwalificatiedossier worden verwerkt in de ontwikkeling van projecten. Dit product ontstaat door een analyse van zowel het huidige als het nieuwe kwalificatiedossier. Vervolgens inventariseer ik de wijzigingen en koppel deze aan de juiste projecten.

Naast het ontwikkelen van de beslisboom richt ik mij op de implementatie ervan. Mijn grootste zorg is of collega’s de beslisboom daadwerkelijk gaan gebruiken. Hoe zorg ik ervoor dat deze verandering succesvol wordt doorgevoerd en tijd en energie bespaart voor zowel mij als mijn collega’s? Daarom bevat mijn beroepsproduct niet alleen de beslisboom zelf, maar ook een implementatiestrategie om deze als tool te integreren tijdens de transitie naar de nieuwe onderwijsvorm.

Mijn onderwijsteam bestaat uit negen docenten, waarvan ik de jongste ben. Binnen vijf jaar zullen vier collega’s met pensioen gaan. Het team heeft de afgelopen jaren veel veranderingen doorgemaakt en staat onder hoge druk. De studentenaantallen zijn de afgelopen vier jaar sterk afgenomen. Daarnaast hebben we in korte tijd vier verschillende managers gehad, wat voor onrust heeft gezorgd.

Binnen het team bestaat een kloof tussen de niveau 3- en niveau 4-opleidingen, wat leidt tot verschillende visies op onderwijs. Hoewel niemand expliciet tegen verandering is, merk ik dat sommige collega’s moeite hebben om hierin mee te gaan. Het team bestaat voor 78\% uit mannen (7 van de 9). Ondanks deze uitdagingen zijn collega’s zeer betrokken bij de studenten en hebben ze hart voor het onderwijs.

Ik kies voor een beslisboom als beroepsproduct omdat dit een gestructureerde en overzichtelijke manier biedt om de veranderingen binnen het nieuwe kwalificatiedossier te integreren in projectmatig onderwijs. De overstap van vakgericht naar projectmatig onderwijs vraagt om een duidelijke structuur, zodat alle leerdoelen correct worden verwerkt in de projecten. De beslisboom helpt collega’s systematisch te bepalen of de nieuwe kwalificatie-eisen op de juiste manier zijn verwerkt in hun projectontwerpen.

Design thinking is de methode die ik ga toepassen tijdens de uitvoering van de beroeps profilering. Ik heb dit verdeeld in de volgende onderwerpen:
Empathize (Empathie ontwikkelen)

    Ik ga in gesprek met collega’s om te begrijpen hoe zij omgaan met het nieuwe kwalificatiedossier en wat hun behoeften en bezwaren zijn.
    Ik onderzoek waarom er weerstand kan zijn tegen het gebruik van een beslisboom.

Define (Probleem definiëren)

    Ik formuleer de kernuitdaging: "Hoe kan ik een beslisboom ontwerpen die docenten helpt de veranderingen in het kwalificatiedossier efficiënt toe te passen zonder dat het als betuttelend wordt ervaren?"

Ideate (Ideeën genereren)

    Ik brainstorm over mogelijke vormen en structuren van de beslisboom.
    Ik betrek collega’s bij het proces om samen een werkbare oplossing te vinden.

Prototype (Prototype maken)

    Ik maak een eerste versie van de beslisboom.
    Ik test deze bij een kleine groep collega’s en verzamel feedback.

Test (Testen en itereren)

    Op basis van feedback pas ik de beslisboom aan en verbeter ik deze.
    Ik implementeer de beslisboom en observeer hoe collega’s ermee werken.
\subsubsection{Pakket van eisen}    
1. Inhoudelijke Eisen

    De beslisboom moet de wijzigingen in het nieuwe kwalificatiedossier volledig en correct verwerken.
    De beslisboom moet aansluiten bij de projectmatige onderwijsstructuur die de school hanteert.
    De beslisboom moet inzichtelijk maken waar en hoe de veranderingen binnen de projecten worden toegepast.
    De beslisboom moet voldoen aan onderwijskundige principes zoals duidelijke leerdoelen en studentgerichtheid.

2. Functionele Eisen

    De beslisboom moet intuïtief en eenvoudig te gebruiken zijn voor docenten.
    De structuur moet logisch en stapsgewijs te doorlopen zijn.
    Er moeten duidelijke keuzemomenten in zitten die docenten helpen bepalen hoe ze een verandering moeten implementeren.
    Het product moet direct bruikbaar zijn in de onderwijspraktijk zonder extra uitleg.

3. Technische Eisen

    De beslisboom moet digitaal beschikbaar zijn (bijvoorbeeld in een PDF of interactieve online tool).
    De beslisboom moet gemakkelijk aanpasbaar zijn bij toekomstige wijzigingen in het kwalificatiedossier.
    Het formaat moet compatibel zijn met de bestaande digitale systemen binnen de school (bijv. Teams, SharePoint of Moodle).

4. Gebruiksvriendelijkheid & Draagvlak

    De beslisboom mag niet als betuttelend worden ervaren door collega’s.
    Er moet ruimte zijn voor feedback en aanpassingen vanuit het docententeam.
    Het product moet laagdrempelig geïntroduceerd kunnen worden binnen het team.
    Er moet een korte handleiding of introductie zijn om docenten snel wegwijs te maken.

Tijdens de totstandkoming van mijn beroepsproduct zal ik actief en voornamelijk interactief gaan werken met onze onderwijskundigen. Daarnaast zet ik mijn collega's in waarvan ik weet dat zij mijn beslisboom willen gaan gebruiken. Mijn docent Alwin Truin en mijn studiecoach Saskia Baars zal ik meenemen bij de voortgang van mijn beroepsproduct. 